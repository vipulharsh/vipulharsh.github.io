%%%%%%%%%%%%%%%%%%%%%%%%%%%%%%%%%%%%%%%%%
% Medium Length Professional CV
% LaTeX Template
% Version 2.0 (8/5/13)
%
% This template has been downloaded from:
% http://www.LaTeXTemplates.com
%
% Original author:
% Trey Hunner (http://www.treyhunner.com/)
%
% Important note:
% This template requires the resume.cls file to be in the same directory as the
% .tex file. The resume.cls file provides the resume style used for structuring the
% document.
%
%%%%%%%%%%%%%%%%%%%%%%%%%%%%%%%%%%%%%%%%%

%----------------------------------------------------------------------------------------
%	PACKAGES AND OTHER DOCUMENT CONFIGURATIONS
%----------------------------------------------------------------------------------------

\documentclass{resume} % Use the custom resume.cls style

\usepackage[left=0.4 in,top=0.4in,right=0.4 in,bottom=0.4in]{geometry} % Document margins
\newcommand{\tab}[1]{\hspace{.2667\textwidth}\rlap{#1}} 
\newcommand{\itab}[1]{\hspace{0em}\rlap{#1}}
\name{VIPUL HARSH} % Your name
\address{web.engr.illinois.edu/$\sim$vharsh2} % Your address
%\address{123 Pleasant Lane \\ City, State 12345} % Your secondary addess (optional)
\address{+1 217 751 2907 $\bullet$ vharsh2@illinois.edu}  % Your phone number and email

\begin{document}


%----------------------------------------------------------------------------------------
%	EDUCATION SECTION
%----------------------------------------------------------------------------------------

\begin{rSection}{Education}

{\bf Masters, Computer Science} \hfill {August 2015 - May 2017 (Expected)}
\\ 
University of Illinois at Urbana-Champaign (3.81/4)

{\bf B.Tech. (Hons.), Computer Science} \hfill {July 2011 - May 2015}
\\ 
Indian Institute of Technology, Bombay (9.18/10)

%Minor in Linguistics \smallskip \\
%Member of Eta Kappa Nu \\
%Member of Upsilon Pi Epsilon \\


\end{rSection}
%----------------------------------------------------------------------------------------
%	TECHNICAL STRENGTHS SECTION
%----------------------------------------------------------------------------------------


%----------------------------------------------------------------------------------------
%	WORK EXPERIENCE SECTION
%----------------------------------------------------------------------------------------

\begin{rSection}{PROJECTS}

\begin{rSubsection}{Collective Operations in charm}{March 2016 - Present}{}{} 
\item Working on improving communication latencies for collective operations (like gather, scatter) in charm. 
\item Improving multicast functionalities in charm for MPI like sub-communicator sections 
 
\end{rSubsection} 


%------------------------------------------------

\begin{rSubsection}{Preventing Overfitting in Machine Learning Classifiers}{Fall 2015}{}{}
\item Explored the properties of Dropout by modifying the training algorithm and analyzed change in performance
\item Also extended Dropout method for neural networks to other learning classifiers like Perceptron and Support Vector Machines
\end{rSubsection}



%------------------------------------------------

\begin{rSubsection}{Improving communication latencies in charm++ using onesided operations }{March 2016 - Present}{}{}
\item Worked on reducing message sending times for charm by having an API that avoids copies for large messages
\item Achieved upto $37\%$ improvement for large messages on BG/Q machines
\end{rSubsection}



\begin{rSubsection}{Parallel version of Floyd Washell algorithm for finding all pair shortest paths in a graph}{Fall 2015}{}{}
\item Analysed and argued the extent of parallelisation possible in the algorithm
\item Predicted performance models and isoefficiency functions for different agglomeration schemes and compared them with experiments on a cluster
\end{rSubsection}





\end{rSection} 


%	EXAMPLE SECTION
%----------------------------------------------------------------------------------------

\begin{rSection}{Experience} \itemsep -3pt  

{Graduate Research Assistant, Parallel Programming Lab (PPL) , UIUC.} \hfill August 2015 - present \\ 
{Summer Research Intern, Georgia Tech.} \hfill May 2014 - July 2014\\
{Research Internship, LaBRI, France
} \hfill May 2013 - July 2013 \\
\end{rSection} 



%----------------------------------------------------------------------------------------
\begin{rSection}{Achievements} \itemsep -1pt {} 
\vspace{-0.4 cm}
\begin{rSubsection}{}{}{}{}  
\item All India Rank $49$ in IIT-JEE $2011$, among $500,000$ candidates
\item Represented IIT Bombay at the prestigious ACM ICPC World Finals 2015. Highest ranked team from India
\item Rank $1$ in $3^{rd}$ International Mathematics Olympiad, $2009$ conducted
by Science Olympiad Foundation
\item Certified as among Top $1\%$
($300$ students) in India, to appear for the following
Indian National Olympiads: Maths (INMO) $2011$; Astronomy (INAO) $2009$, $2011$
 \end{rSubsection}
\end{rSection}



\begin{rSection}{Technical Skills}

\begin{tabular}{ @{} >{\bfseries}l @{\hspace{6ex}} l }
Programming & C, C++, Java, Python, MPI, Matlab, Charm++\\ 
Scripting & Bash, Slurm, pyplot, HTML \\  
Miscellaneous &  $\LaTeX$, Scheme, Prolog \\
\end{tabular}

\end{rSection}





\end{document}
